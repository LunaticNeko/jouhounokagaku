\documentclass{article}
\usepackage{xeCJK}
\setCJKmainfont{MS Mincho} % for \rmfamily
\setCJKsansfont{MS Gothic} % for \sffamily
\usepackage{hyperref}
\usepackage{lscape}
\usepackage{longtable}
\usepackage{multirow}
\usepackage{xcolor}

\title{Computer Science and Information Processing \newline
[情報の科学] in English}
\author{Asst. Prof. Chawanat \textsc{Nakasan}, Information Media Center}
\date{}

\begin{document}
\maketitle

\noindent
NOTE: This ``hard copy" syllabus contains extended information over the web version. However, where content conflicts, the web version takes precedence.

\section{授業の主題 Subject of Class}
This class is a series of lectures on computers, problem solving, algorithms, artificial intelligence, networks, database, and information security.
The class is conducted fully in English.

\section{授業目標 Class Objective}
The objective of this class is to provide fundamental knowledge about computers, problem solving, algorithms, artificial intelligence, networks, database, and information security.
In particular, problem solving and algorithm skills will also be taught with practicality in mind.

\section{学生の学修目標 Student's Learning Goals}
You are expected to understand what is covered in the class objective, including fundamentals of computers and their components, problem solving, algorithms, artificial intelligence, networks, database, and information security. During the class, you are expected (but not required) to expand your knowledge using outside learning sources beyond the main textbook. Gaining appreciation for computer science would be a bonus.

By the end of this course, you should be familiar with various concepts in computer science and also be ready for advanced classes in the same field.

\section{学修成果 Learning Outcomes}
\label{goals}
Please read this section carefully as the following list describes contents of the class and what you are required to learn. You will be tested precisely about these topics.

After taking this class, you should be able to:
\begin{enumerate}
    \item Explain the basic architecture and components of a computer, as well as each component's role.
   	\item Explain and analyze algorithms, and draw flowcharts from problems.
   	\item Make a basic comparison of asymptotic computational complexity using big-O notation.
   	\item Solve basic problems representative of common algorithms.
   	\item Explain basic concepts of artificial intelligence and use it to address computational problems.
   	\item Describe all seven layers of OSI Model and five layers of TCP/IP Model and datagram encapsulation process.
   	\item Explain functions of computer networks, especially regarding important protocols.
   	\item Solve relational algebra operations (union, join, and so on).
   	\item Explain basic information security keywords, concepts, measures, tools, and what to do in an emergency.
   	\item Demonstrate safeguarding of personal information security and privacy.
\end{enumerate}

\section{授業概要 Content}
The course is broken into multiple distinct parts. Most content is based on J\={o}h\={o} no Kagaku (Japanese) and ``Information Science'' (English) textbooks. Programming is not required, but may be utilized to your advantage.

\begin{enumerate}
    \item Week 1: Computer and Architecture: definition of a computer; components of a computer; arithmetic and logical operations
    \item Week 2-3: Problem Solving and Algorithms: flowcharts; computer program flows (branches and loops); algorithms; time complexity and their analysis; asymptotic analysis (big-O Notation); concepts of problem solving; data structure; demonstration of algorithms
    \item Week 4: Artificial Intelligence (AI): optimization; genetic algorithm (GA); machine learning (ML); neural networks (NN); deep learning (DL); familiarization with various concepts in AI; examples of AI in real life
    \item Week 5: Computer Networks: data communication; network topology; OSI model; TCP/IP model; encapsulation; network interface; switching and routing; data transport; UDP; TCP; DNS; HTTP
    \item Week 6: Databases: relational databases; operations in relational databases; normalization; database management system (DBMS)
    \item Week 7: Information Security and Computer Ethics: concepts and keywords; CIA (confidentiality, integrity, availability); personal security measures (passwords, two-factor authentication, antivirus software, firewalls); privacy; data encryption; computer crimes; real-world case studies
    \item Week 8: review and free discussion (examination if held)
\end{enumerate}

\section{評価方法 Evaluation Method}
Evaluation is based on in-class participation, quizzes, two reports, and a final examination (if held).

\smallskip\noindent
\textbf{Participation}: Students are highly encouraged to participate in class by asking and volunteering for class activities. These activities can be conducted online, such as discussion, presentation, and student-led tutorials. Students are encouraged to discuss with the instructor about possible methods of participation. Although the class will be principally held face-to-face, attendance is not required.

\smallskip\noindent
\textbf{Quizzes}: There are quizzes at the beginning (or end, if online) of each class. Questions based on previous content are graded. Questions marked ``pre-test" are not graded. You must sincerely attempt all questions to have your quiz scored.

\smallskip\noindent
\textbf{Grade Letters}: Correspondence between total score and grade is based on University regulation [S, A, B, C] $\rightarrow$ [90\%, 80\%, 70\%, 60\%] respectively.

\section{評価の割合 Evaluation Breakdown}
If the final examination cannot be held, the breakdown in the parentheses will be used.

\begin{itemize}
    \item 5\% (10\%) Participation and Quizzes (see paragraph below for details)
    \item 20\% (45\%) Mid-Term Report
    \item 25\% (45\%) Final Report
    \item 50\% (0\%) Final Examination
\end{itemize}

For the Participation and Quizzes part, if the student does not participate in classroom activity, only quiz scores will be used. However, active participation will reward participation scores that can be used to replace missed quizzes. \textcolor{red}{This is done by adding both Participation and Quiz scores together, but only up to 5\% (10\%) is applied towards your final grade. Within Participation, 1\% is awarded for completing all questionnaires.}

\textcolor{blue}{Due to the ongoing novel coronavirus (COVID-19) outbreak, the final examination may be cancelled at the instructor's discretion. In this case, evaluation will be principally based on the reports.}

\section{予習に関する指示 Class Preview}
Students are recommended to prepare for each class beforehand using available learning materials. Slides and commentaries of the textbook (in English) will be made available as they are ready.

\section{復習に関する指示 Class Review}
Students are required to study after the classes throughout the semester. Quiz questions based on previously learned content are \textbf{graded}.

\begin{landscape}

\section{ルーブリック Rubrics}

\begin{longtable}{|p{3.0cm}|p{4cm}|p{4cm}|p{4cm}|p{4cm}|}
\hline
Area & \multicolumn{4}{c|}{Grading Level} \\
\hline
 & Excellent (90\%) & Very Good (80\%) & Good (70\%) & Average (60\%) \\
\hline\hline
\endhead
1 Computer Architecture and Components &
  Can describe computer architecture and their components in detail, giving examples and comparisons for all categories. &
  Can give a general description of computer architecture and their components, giving some examples and comparisons. &
  Can give a basic description of computer architecture and their components or give some examples. &
  Can name various parts of the computer and their components.
  \\
\hline
2 Algorithms and Flowcharts &
  Can accurately translate between word problems, algorithms, and flowcharts. &
  Can make simple translation between word problems, algorithms, and flowcharts. &
  Can consistently understand word problems, algorithms, and flowcharts. &
  Can understand some word problems, algorithms, and flowcharts.
  \\
\hline
3 Asymptotic Analysis &
  Can analyze problems and give accurate asymptotic analysis. &
  Can prove computational complexity of simple worked algorithms. &
  Can explain and compare each level of asymptotic notation. &
  Can name a few asymptotic notations for some representative problems.
  \\
\hline
4 Problem Solving in Algorithms &
  Can consistently solve competitive-level problems or otherwise shows competitive potential.&
  Can solve most lecturer-introduced in-class problems. &
  Can independently solve some of the harder in-class problems. &
  Can consistently solve the easier or partially-worked harder in-class problems.
\\
\hline
5 Artificial Intelligence (AI) &
  Can consistently explain concepts in AI and how they can be used to solve some representative problems. &
  Can explain concepts in AI and some of their general benefits and limitations, and model simple problems. &
  Can explain about AI and its nature of operations. &
  Can briefly explain about general concepts of AI.
\\
\hline
6 OSI and TCP/IP Models &
  Can accurately explain packet encapsulation process and list all layers of OSI and TCP/IP Models and their functions. &
  Can explain packet encapsulation process and list all layers of OSI and TCP/IP Models and their functions. &
  Can provide examples of packet encapsulation and name (but not list) all layers of OSI and TCP/IP Models and their functions. &
  Can briefly explain about packet encapsulation and can correlate the ``number" of each OSI and TCP/IP Model layer to their functions.
\\
\hline
7 Network Functions and Protocols &
  Can explain all network functions and protocols in detail, as well as interpret and construct datagrams of each protocol. &
  Can explain most network functions and protocols, as well as show general understanding of each protocol's datagrams. &
  Can explain some network functions and protocols, and list each protocol's key data fields. &
  Can answer basic questions about network functions and protocols.
\\
\hline
8 Relational Database Theory &
  Can perform complex (nested) operations with multiple operators. &
  Can perform simple (unnested) operations with multiple operators. &
  Can perform operations with one operator at a time. &
  Can perform operations only with clear guidance and instruction for each problem.
\\
\hline
9 Information Security and Ethics &
  Can consistently explain concepts of information security and computer ethics. &
  Can generally explain concepts of information security and computer ethics. &
  Can somewhat explain concepts of information security and/or computer ethics. &
  Can briefly explain information security and/or computer ethics.
\\
\hline
10 Personal Information Security &
  Can take practical actions to safeguard personal information. &
  Can independently make efforts to safeguard personal information. &
  Can safeguard personal information after some urging. &
  Can briefly explain the importance of \textit{personal} information security.
\\
\hline

\end{longtable}

\end{landscape}

\section{教科書 Textbook}
The class is entirely based on the J\={o}h\={o} no Kagaku text. 

\url{http://ilas.w3.kanazawa-u.ac.jp/students/subject/gs/gs_text/}

This class uses real-world competitive programming problems and data sets. Therefore, you may be asked to download additional data files to complete exercises. Where possible, the lecturer will attempt to provide them for convenience.

\section{参考書 Reference Books}
No reference books are required.

\section{教科書・参考書補足 Supplementary Text and Reference Books}

\section{その他履修上の注意事項や学習上の助言 Other Guidelines on Study and Learning}
Students are highly advised to pursue topics of interest on their own using both online and offline sources. As for the report, students are recommended to work in groups of similar interests. As long as the reports are written individually, the work unit (programs, etc.) can be shared.

Students are also advised to learn a simple programming language such as JavaScript and Python in their own time.

\section{オフィスアワー等(学生からの質問への対応方法等) Office Hours and Inquiries}
Please contact by e-mail. See instructor information for more details.

\section{履修条件 Course Conditions}
As an introductory class intended for freshmen and audience new to computer science, there are no prerequisites (courses that must be taken prior) to this class. However, it is advisable that students have prior knowledge in discrete mathematics and precalculus concepts. Programming is not required but may be used in your reports. Basic computer operation skills (connecting to a network, word processing, using spreadsheets, sending a file, registering a new account for an online service, etc.) will be a plus. These skills are taught in Information Processing Basics (情報処理基礎) and/or Data Science Fundamentals (データサイエンス基礎) classes, but they are not hard prerequisites (required) for this class.

\textcolor{blue}{Due to the novel coronavirus outbreak, the class may be switched to online and you are expected to be familiar with online teleconference and productivity tools. Specific teleconference tool will be determined after confirming student count. You will be given instructions on how to sign up or activate the tool, as well as the instructor's username or email address for teleconferencing before the first class.}

\section{適正人数 Expected Number of Students}
20

\section{受講者調整方法 Student Adjustment Method}
This class pursues a more active approach and discussion will greatly benefit all participants. There are rarely, if any, wrong answers or ways to approach a problem. Please do not be afraid to participate.

The class itself adapts to student numbers. Some of the class content and report themes will be based on the students' home faculties and departments.

\section{カリキュラムの中の位置づけ Position in Curriculum}
\section{特記事項 Notices}
This section is not included in the web version.

\subsection{Special Classroom Rules (``The Rules'')}

\textbf{Protect Yourself}: As mentioned in [\ref{goals}], students are expected to exercise good information security practices. Those caught with substandard information security behavior will receive a verbal warning, then a penalty (up to 2\% of final score) against their total grade. \textit{This will not push you from C to fail.}

\smallskip\noindent
\textbf{Plagiarism}: All your work must be your own. This class has zero-tolerance policy on plagiarism. Mechanical and human methods will be employed to detect plagiarism and all offenses are subject to disciplinary action. You are encouraged to help each other with the reports, but the contents on the paper must be your own. When you are unsure, it is safer to provide references.

\smallskip\noindent
\textbf{Active Learning}: In-class participation is expected. Students are free to bring up related topics for discussion. Students are welcomed to voice their opinions directly or through e-mail at any time.

\subsection{Reports}

Themes will be announced during the class. Each report (mid-term and final) should be about 2-4 pages long, and submitted using the LMS in PDF (only). Paper setting should be A4 (A3 accepted for pages containing solely diagram or database schema). For programming reports, source code files must be attached separately and do not count towards page limit. All reports should use 10 to 12-point font for main text, with adequate line (at least single) and paragraph (at least half-line) spacings (Note: Default settings in modern word processors and leaving one blank line between paragraphs will suffice). Ensure that you have included your name and student ID. If you have a name in kanji, please include both kanji and English.

If you would like to also make a presentation to the class, please let me know during report theme announcement. I will give you some time to do so (subject to time constraints).

\smallskip\noindent
\textbf{Late Policy}: If your report is late, you will lose 20\% of total credit earned for that report per working day. Reports five days late will not earn any credit. Late quizzes will not be graded and you will lose participation bonus for completing all quizzes. If there are special circumstances, please let me know.

\smallskip\noindent
\textbf{Report Attachments}: If you wish to use programming in a report, please include the source code \textit{and indicate the compilation method} as attachment. The instructor will use best-effort otherwise and will not grade your programs if deemed impossible. Images should be of reasonable size and format. File archives should be .zip, .gz, or .7z (.rar is not recommended). Binary executables and embeds (such as EXE files and MS Office macros) are not accepted. MS Office Macro-enabled files (.docm, .xlsm, .pptm) are automatically rejected even with no macros inside. Please be careful when saving.

\smallskip\noindent
\textbf{Multiple Submissions}: If you submit many reports (or quizzes, in online classes), only the latest one (before the deadline) is considered.

\subsection{Language}

All content and activities in this class will be in English. I will do my best to explain the quiz problems and report requirements in Japanese if required. If you require any further accommodation (large font, colorblind friendly, use of hearing aid, speech recognition friendly, etc.), please let me know.

\subsection{Special Considerations due to COVID-19}

Due to COVID-19, special considerations and measures are being taken.

\textbf{Reduced Contact}: The instructor has redesigned some class activities to have less physical contact while retaining learning experience. We will be limiting number of classroom occupancy (according to University policy).

\textbf{Alternative Activities}: Although the class is best attended in person, the instructor will not penalize absent students. Participation points can be earned by active discussion in the class forums or chatroom (via university LMS or other means) or presentation of interesting content or your report (as a recorded video, by pasting links to personal blogs, or other means). However, if the examination will be held, this examination must be taken in-person (except extenuating circumstances).

You are requested to help prevent the spread of COVID-19 by following classroom policies.

\section{キーワード Keywords}
computer, computer architecture, information networks, computer networks, information processing, problem solving, algorithms, artificial intelligence, data communication, database, SQL, computer security, network security, information security

\section*{Instructor Information}
Asst. Prof. Chawanat \textsc{Nakasan}, D.Eng.

\smallskip\noindent
Office: Information Media Center (Building Number C-2), 2F, Room A202

\smallskip\noindent
Hours: Available most Tuesdays. Please email me ahead of time for live online consultations. Direct visit is not recommended unless your report/project is physical in nature (e.g. IoT, hardware development).

\smallskip\noindent
Office Phone: 076-234-6928

\smallskip\noindent
E-Mail: \{\textit{firstname}\}@staff.kanazawa-u.ac.jp

\end{document}
